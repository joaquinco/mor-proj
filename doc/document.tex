\documentclass{article}
  \parindent = 0mm % Sin sangría
  \usepackage[utf8]{inputenc}
  \usepackage[T1]{fontenc}
  \usepackage[spanish]{babel}
  \usepackage{graphicx}
  \usepackage{amstext}
  \usepackage{amsmath}
  \usepackage{booktabs}
  \usepackage{subfigure}
  \usepackage{footnote}
  \usepackage{hyperref}
  \usepackage{algpseudocode,algorithm,algorithmicx}
  \usepackage[font=small,labelfont=bf]{caption}

\begin{document}
  \begin{center}
    {\sc \large Metaheurístcas y optimización sobre redes}
    
    {\sc \large Obligatorio 2020}
    \linebreak

    {\rm Joaquín Correa - \today}
  \end{center}

  \section*{Introducción}

  En este documento se presenta GRASP como metaheurístca para el problema de ruteo de vehículos con ventanas de tiempo y flota heterogénea. Como variante, se agrega la posibilidad de ventanas de tiempo flexibles, es decir permitir a los vehículos llegar a los clientes luego de la ventana de tiempo predefinida, pagando cierta penalidad.

  \section{*Problema}

  Formalmente, el problema se define como:
  TODO: No se si da para redefinir el problema matematicamente. Si da para explicarlo y referenciar el paper de Caludio.

  \section*{Metaheurístca}

  TODO: Por que se eligió GRASP. ventajas/desventajas que pueda tener en este problema frente a otros.
  Ej: tabu ya se uso, evolutivos/genetico no daba (o ya se estaba haciendo en otro curso)

  \section*{GRASP}

  TODO: características del GRASP

  \subsection*{Fase de construcción}

  \subsection*{Búsqueda local}

  \section*{Referencias}

  \begin{enumerate}
    \item{ \label{repo} \url{https://github.com/joaquinco/mor-proj}}
  \end{enumerate}
\end{document}
